\chapter{Conclusions} % Write in your own chapter title
\label{Chapter6}
\lhead{Chapter \ref{Chapter6}. \emph{Conclusions}} % Write in your own chapter title to set the page header

This thesis has presented a large scale, geographically distributed collaborative application which is deployed on a group of clouds. In order to achieve better QoS while minimizing the cost of running the application the system can be endowed with a self-organizing, autonomic system responsible for managing the application. The thesis has presented two algorithms for such a self-organizing system - one to detect the SLA breach and one to optimize the number of servers in the cloud. A test bed was also created to run the application and validate the control system. Results were obtained, which prove that the control of the system can not be achieved by simply managing the CPU usage of the servers and that a more complex system, like the one presented in the thesis has to be used.

The novelty of the thesis comes from three separate parts which support each other:

\begin{enumerate}
	\item The design of the geographically distributed collaborative media cloud. This architecture is the first architecture which makes use of the client location to connect clients to a clouds closer to them, while at the same time allowing clients to communicate with each other as if they were connected to the same server. Normally clients would all connect to the same server in order to interact with each other as seen in various multiplayer games for example.
	\item The ACO algorithm used in order to detect breaches of SLA. Normally the ACO is used in order to optimize some form of problem, however the thesis uses a modified version of the ACO in order to analyze the behaviour of the cloud, detect breaches of SLA and take corrective measures.
	\item A new self-organizing algorithm based on the behaviour of ants while house hunting is implemented and experiments to validate it are introduced. The algorithm is inspired by prior work and modeling into the behaviour of ants while house hunting, however there is no known algorithm which implementats the behaviour in order to optimize a given problem.
\end{enumerate}

The thesis opens a number of avenues for future work and improvements to the new algorithms introduced:

\begin{enumerate}
	\item When new servers are added to a cloud, it would be desired to move some of the load from one server to another. This should be done in such a way such that a client would not notice that his server has changed. A mechanism to allow such a move for a client can be developed in the future.
	\item The ACO SLA detection algorithm uses a single value being fed from the model to decide how much pheromone an ant should add. However, to improve stability in the cloud the system should use a range of values to decide if the amount of pheromone to be added should match the decaying values such that the cloud stays stable. With a single value, small deviations in the model can cause over long periods of time new servers to be added or servers to be removed.
	\item The house hunting algorithm can be improved to take in consideration the fitness of a solution when recruiting. While the fitness should not be the only factor in deciding which ant should recruit another, it can be used to bias positively solutions with better fitness. The current solution uses the fitness only to remove solutions with very poor fitness from the solution set.
\end{enumerate}